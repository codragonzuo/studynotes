```tex
\documentclass[11pt,a4paper]{article}
\usepackage[UTF8]{ctex}
\usepackage[landscape]{geometry}
\usepackage{ucs}
\usepackage{amsmath}
\usepackage{amsfonts}
\usepackage{amssymb}
\usepackage{graphicx}
\usepackage{textpos}
\begin{document}
	90中午
	\pagestyle{empty}
	
\begin{textblock*}{4cm}[2,3](10cm,15cm)
身份证
\end{textblock*}

\begin{textblock*}{4cm}[2,3](5cm,15cm)
	身份证
\end{textblock*}
\end{document}
```


There is an alternative, starred, form of the {textblock} environment. In the textblock*
argument to the {textblock*} environment, the block width, and the block po-
sition (but not the specification of the block reference point) are given as absolute
dimensions, rather than as numbers in units of the horizontal and vertical modules.
Thus
```
\begin{textblock*}{hhsizei}[hhoi,hvoi](hhposi,hvposi)
text...
\end{textblock*}
```
produces a textblock of the given size, where this time hhsizei, hhposi and hvposi
are absolute dimensions, but hhoi and hvoi are still pure-number offsets (that is,
fractions of the width and height of the textblock), as above.

Each {textblock} environment takes up zero space on the page (which means,
by the way, that it cannot detect that it’s overprinting or being overprinted), so
you can (and typically will) use several of the environments in a row to scatter
text all over the page.

The package is compatible with the calc package, so that you may use calc-style
expressions when specifying lengths. Thus

```
\usepackage{calc}
\textblockorigin{56.9055pt-10mm}{0pt+1cm}
\begin{textblock*}{10mm+14cm}(0.3cm*5,10\TPVertModule+5mm)
6text. . .
\end{textblock*}
```

Note that you can only use calc-style expressions where you would specify a length
with units, such as the width and location arguments of {textblock*} or the
arguments to \textblockorigin – you can’t use them when specifying a length
in units of the horizontal and vertical modules, such as in the width and location
arguments to the (unstarred) {textblock} environment.
